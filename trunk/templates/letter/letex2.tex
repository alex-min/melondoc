% This file is part of the "lettre" package.
% This package is distributed under the terms of the LaTeX Project 
% Public License (LPPL) described in the file lppl.txt.
%
% Denis M�gevand - Observatoire de Gen�ve.
%
% Ce fichier fait partie de la distribution du paquetage "lettre".
% Ce paquetage est distribu� sous les termes de la licence publique 
% du projet LaTeX (LPPL) d�crite dans le fichier lppl.txt.

\documentclass[12pt,origdate]{lettre}
\usepackage[francais]{babel}
\usepackage[OT1]{fontenc}
\usepackage{mltex}
\begin{document}

%
% Entete et signature par defaut, format plain.
% =============================================
% Champs: objet, copies, annexes, post-scriptum.
% ==============================================
%
\begin{letter}{	Pr.~E.T.~Phonom \\ 
                D\'epartement d'Asprototographie \\ 
                Universit\'e de Saint Zopium \\
                3945, Quai du G\'eneral Gisant \\
                CH-6800 Motte-au-Rolla }

\pagestyle{plain}

\name{Dr~S.~E.~Dnavegem}

\conc{Sixi\`eme Symposium Al\'ea\-toi\-re Intercommunal
de Dynamotoculture (~SAID~1993~)} 

\opening{Cher Professeur Phonom,}

Je vous remercie d'avoir donn\'e suite \`a ma requ\^ete, et vous
confirme ma participation au symposium en tant que sp\'ecialiste
des affaires \'etranges. 

\closing{Veuillez agr\'eer, Monsieur le professeur, l'expression
         de mes condol\'eances distingu\'ees.} 

\cc{Pr.~Zoldan Fratschski \\
    Me  Barillada \\
    Ra\"{\i}ssa Goba }

\encl{Talon de participation \\
      Bons de visite (6) \\
      Article Dnavegem }

\ps{PS :~}{Veuillez trouver en annexe les documents dont nous avons
parl\'e ce matin au t\'el\'ephone, ainsi qu'un exemplaire de mon
article consacr\'e \`a la culture intensive du Yen Japonais en
milieu hospitalier. Je vous en souhaite bonne r\'eception.} 

\end{letter}
%
\end{document}
